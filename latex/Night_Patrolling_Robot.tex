\documentclass[journal]{IEEEtran}
\usepackage{cite}
\usepackage{amsmath,amssymb,amsfonts}
\usepackage{algorithmic}
\usepackage{graphicx}
\usepackage{textcomp}
\usepackage{xcolor}
\usepackage[all]{nowidow}
\usepackage[none]{hyphenat}

% \usepackage[switch  ]{lineno}

\def\BibTeX{{\rm B\kern-.05em{\sc i\kern-.025em b}\kern-.08em
T\kern-.1667em\lower.7ex\hbox{E}\kern-.125emX}}

\begin{document}

% \linenumbers

\title{Night Patrolling Robot with Facial Recognition}

\author{
	\IEEEauthorblockN{
		Manisha Dhage\IEEEauthorrefmark{1}
		Vivek Ghuge\IEEEauthorrefmark{2},\\
		Divija Godse\IEEEauthorrefmark{3},
		Mitrajeet Golsangi\IEEEauthorrefmark{4},
		Pravin Harne\IEEEauthorrefmark{5},
	}
	\IEEEauthorblockA{
		\textit{dept. of Computer Science} \\
		\textit{Vishwakarma Institute of Technology}\\
		Pune, India\\
		Email : \IEEEauthorrefmark{1}manisha.dhage@vit.edu,
		\IEEEauthorrefmark{2}vivekghuge2002@gmail.com,
		\IEEEauthorrefmark{3}divijagodse@gmail.com,
		\IEEEauthorrefmark{4}mitrajeetgolsangi@gmail.com,
		\IEEEauthorrefmark{5}sunnyharne008@gmail.com,
	}
}

\maketitle

\begin{abstract}
	Abstract Here
\end{abstract}

\begin{IEEEkeywords}
	Keywords
\end{IEEEkeywords}

\section{Introduction}

\section{Literature Review}
One of the most important needs of modern times is
security. The rise in population is directly proportional
to the need for proper security. A night patrolling
visionary robot would help achieve certainty, especially
during the night. All the research papers have directed
to the fact that the basic must-haves for the night
patrolling systems are the logic for sound sensing,
moving towards the area of target and back to the
original location, image capturing and processing.\\

The night vision patrolling robot has the primary
objective to ensure the safety of the people without
putting any human life at risk. There are certain
essential features needed in the robot, such as a night
vision, obstacle detection, and motor driver circuits
for controlling them\cite{[1]}. All of these are controlled
using a microcontroller board such as an Arduino UNO
or raspberry pi. In addition, a wireless IR transceiver
would be helpful in the navigation of the robot\cite{[3]}.
According to \cite{[1]} the proposed system has the best
advantage over disadvantage ratio if it uses an IR
Camera with a wireless controller and some form of
an obstacle detection system. Furthermore, optional
features such as a GPS Module, GSM radios \cite{[3]}, sound
sensor\cite{[1]} will greatly help increase the productivity
of the project.
Journals have shown that a sound sensor and a smart
camera are primary necessities for the system. Along
with this according to the paper by N.Hemavathi the
robot is built with a DC motor and transistors.
The movement of the robot is handled by multiple
logics operated on the transistors. These transistors
allow the motion of the DC motor in the desired
direction\cite{[6]}. Bluetooth technology has also been
used in the process for serial communication with
the robot\cite{[7]}. In another paper, the authors have
mentioned the use of a special GPS for location tracking.
Object detection algorithms have been designed for the
robot to understand the route\cite{[8]}.
Technology is advancing at a rapid pace nowadays. These
shifts are also visible in the robotics industry. As a
result, the system suggested in the given papers reflects
the current state of the field. The main objective of this
is to offer ladies safety at night. Using a credit-card-sized
Raspberry Pi and an open CV, the author of this paper,
created a night patrolling robot (computer vision).\cite{[11]}
To detect the face, the suggested system employs a Raspberry
Pi camera. Anomaly detection is done using deep learning
technologies\cite{[13]}. As a result, the image is recorded using
the pi camera and sent to the raspberry pi for face and
human detection using OpenCV.  The accuracy of this
approach is around 83 percent\cite{[11]}.The sensors used are
IR sensors and sound sensors. According to this project,
the entire territory monitoring is completed using a night
vision camera and a programmed framework. When a sound is
detected, the robot will follow a certain path to that
space, capture the region, and send the picture to the
user via IoT. Behind this project is a pre-programmed
dazzling path for night vision viewing.\cite{[12]}
The proposed system focuses on the security of the place
where it is implemented. It is specially designed to
carry out the security assessment function at the night
time when it will be dark all over the place. It is equipped
with a night vision camera which will be used to capture
the pictures of the spot if any suspicious activity takes
place. It moves on a predestined path, and it moves in
the direction of sound. It is also equipped with human
face recognition technology. This system uses the LAN
protocol of IOT, and it sends the recorded images to the
responsible authority so that they can take any actions
if necessary. It system uses a microcontroller, night
vision cameras, sound sensors, Infrared sensor and Motor
drivers.\cite{[15]}
This system is built to ensure women safety in remote
regions and public places. IOT gecko is used for
receiving transmitted snapshots and displaying them to
people with alert sounds. This system consists of a
combination of 2 HD cameras to monitor the environment
sharply and closely. The hardware used in this system are
HD infrared camera with night vision, Sound sensor,
DC motor, IR Sensor, Ultrasonic sensor, LCD display, and
a motor driver.\cite{[16]}\\

Facial recognition is an important aspect in terms of the
proposed system. This serves a greater importance in order
to assess the threat and act accordingly.\cite{[2]} Facial
recognition is a subset of the highly expanding computer
vision field. This field has been overly dominated by the
areas of Machine Learning and Deep Learning from the very
beginning\cite{[5]}. Taking into consideration the scope of
lighting in a night patrolling robot it is safe to assume
that the best approach would be to perform IR recognition\cite{[2]}.
Thus according to \cite{[2]} the best possible way of training
the TIR based facial recognition is to use a Deep CNN
Classifier with substantial training data. The losses
given by equation 1
\begin{equation}
	\sum_i^N[\| \int (x_i^a) - \int (x_i^p)\|_2^2 - \| \int (x_i^a) - \int (x_i^n)\|_2^2 + \alpha]
\end{equation}
Are the minimum and give the best possible output with a
method called triple loss learning.\cite{[2]} There are
three main steps for facial recognition and tracking in
real time\cite{[4]}. The First Step is Detection of Faces to track
them, the next being recognition of facial features, which is determined
using equation 2
\begin{equation}
	\begin{aligned}
		\Omega^m_{[w, h]} =
		 & \begin{pmatrix}
			[\frac{m}{w}] + [\frac{m - 1}{w}] + \ldots + [\frac{w + 1}{w}] + 1
		\end{pmatrix}
		\cdot                        \\
		 & \begin{pmatrix}
			[\frac{m}{h}] + [\frac{m - 1}{h}] + \ldots + [\frac{h + 1}{h} + 1]
		\end{pmatrix}
	\end{aligned}
\end{equation}
The last step is to begin tracking the face in real time.

Object detection bases the image processing. The image is
seen in a digital form of the same. Process of Image
processing, object detection majorly involves object recognition,
object localisation, image classification. According to a
survey for different processes for image processing,
multiple techniques have been identified which present
different accuracies for the same. The survey paper implies
that numerous techniques have separate sets of specifications.
For example, Deep CNN shows 85\% sensitivity especially for
medical cases, face recognition vendor test (FRVT) makes
identifying a male easier than female. According to this paper,
Deep neural networks and CNN which are AI based techniques are
recommended for object detection \cite{[9]}.  Furthermore, another
paper by Sandeep Kumar, has emphasized on the kind of image
processed. It mentions the use of Convoluted neural networks,
for static images with static background. After preprocessing,
and feature extraction, single neural networks are integrated
for image recognition and processing\cite{[10]}.

The author of this work presents several innovative models for
all stages of a face recognition system. To perform the process
of face detection efficiently, the authors suggest a hybrid
model integrating AdaBoost and Artificial Neural Network (ABANN).
The labeled faces detected by ABANN will then be aligned using
the Active Shape Model and Multi-Layer Perception in the next
stage. The author proposes a novel 2D local texture model based
on Multi-Layer Perceptron in this alignment step. The model's
classifier enhances the accuracy and robustness of local
searching on faces with ambiguous outlines and expression
variation. The authors of the research offer a way for boosting
efficiency in the feature extraction step by combining two
methods: geometric feature-based method and Independent
Component Analysis method.\cite{[14]}

\section{Methodology}

\section{Results and Discussion}

\section{Conclusion}

\pagebreak

\bibliographystyle{IEEEtran}
\bibliography{references.bib}

\end{document}
